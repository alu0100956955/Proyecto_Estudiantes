Dado que necesitamos generar gráficos utilizaremos la herramienta denomina \href{http://www.jfree.org/jfreechart/}{\tt J\+Free\+Chart}. J\+Free\+Chart es un marco de software open source para el lenguaje de programación Java, el cual permite la creación de gráficos complejos de forma simple.

J\+Free\+Chart trabaja con G\+NU Classpath, una implementación en software libre de la norma estándar de biblioteca de clases para el lenguaje de programación Java. J\+Free\+Chart es compatible con una serie de gráficas diferentes, incluyendo cuadros combinados. Estos gráficos son\+:
\begin{DoxyItemize}
\item Gráficos XY.
\item Gráfico circular.
\item Diagrama de Gantt.
\item Gráficos de barras (Histogramas).
\item Single valued (Termómetro, brújula, indicador de velocidad).
\end{DoxyItemize}

\#\#\# Para el uso la librería, se utiliza (mediante \href{https://maven.apache.org/}{\tt Maven}) 
\begin{DoxyCode}
<\textcolor{keywordtype}{dependency}>
    <\textcolor{keywordtype}{groupId}>\textcolor{keyword}{org.jfree}</\textcolor{keywordtype}{groupId}>
    <\textcolor{keywordtype}{artifactId}>\textcolor{keyword}{jfreechart}</\textcolor{keywordtype}{artifactId}>
    <\textcolor{keywordtype}{version}>1.0.14</\textcolor{keywordtype}{version}>
</\textcolor{keywordtype}{dependency}>
\end{DoxyCode}


\subsubsection*{Inicialmente se da divido en tres grandes módulos de desarrollo.}

{\bfseries Entradas}
\begin{DoxyItemize}
\item Vamos a desarrollar una aplicación que tendrá datos de entrada Datos Abiertos (Open Data), tales que los formatos de dichos ficheros van a ser\+:
\begin{DoxyItemize}
\item “.\+A\+S\+C\+I\+I”
\item “.\+cvs”
\item “.\+xls”.
\end{DoxyItemize}
\end{DoxyItemize}

{\bfseries Proceso}
\begin{DoxyItemize}
\item Normalizar y formalizar los datos a un formato comprensible para J\+Free\+Chart.
\item Gestión del procesamiento de la salida.
\end{DoxyItemize}

{\bfseries Salidas}
\begin{DoxyItemize}
\item Gestiona la salida dando la posibilidad de mostrar la información en distintos formatos al usuario, como son\+:
\begin{DoxyItemize}
\item Diagrama de barras
\item Diagrama de pastel
\item Información mediante la consola
\item Imprimir en pdf (imprime la opción elegida de las anteriores)
\end{DoxyItemize}
\end{DoxyItemize}

Para la gestión de la documentación se hace uso de la herramienta \href{http://www.stack.nl/~dimitri/doxygen/}{\tt Doxygen} 